\documentclass{article}
\usepackage{amssymb,amsthm,amsmath,amsfonts}

\newtheorem{theorem}{Theorem}
\newtheorem{definition}{Definition}

\begin{document}

\section{Definitions}

Define a collatz sequence with initial term $T$ by 
\[
C_0=T\textrm{ and }C_{n+1}=\left\{
\begin{array}{cll}\frac{C_n}{2}=\delta(C_n)&\textrm{if }C_n\textrm{ is divisible by }2&A \emph{dividing} iteration\\
3C_n+1=\mu(C_n)&\textrm{otherwise}&A \emph{multiplying} iteration\end{array}\right.
\]
for $n\geq 1$.

\section{Alternative}
This is a sequence from J.H. Conway
\begin{eqnarray*}
f(4n+1)&=&3n+1\\
f(4n-1)&=&3n-1\\
f(2n)&=&3n
\end{eqnarray*}

Note that this can be inverted.  Empirically, this tends to produce sequences going off both ways to $\infty$.

\section{Observations}

Besides the trivial cycle ($0$ repeating), there are four known cycles which may occur in such sequences:
\[\begin{array}{c}
4,2,1,4\\
0,0\\
-5,-14,-7,-20,-10,-5\\
-2,-1,-2\\
-17,-50,-25,-74,-37,-110,-55,-164,-82,-41,-122,-61,-182,-91,-272,-136,-68,-34,-17
\end{array}\]
We are most interested in the cycle of positive numbers.  Most of what follows is specific to initial terms $T>0$.

\begin{definition}
A \emph{glide} is the sequence of collatz iterations performed on an integer $k$ until the result $k_1\leq k$.
\end{definition}

$27$ is the smallest number with a shockingly long glide - of length $97$!

\section{Cross References}
There is a vague connection to the ``On-Line Encyclopedia of Integer Sequences'' sequence A036994.  This is seen when you look at the 
increasing sequence of positive numbers which is left after you filter out all the numbers which have a redundant glide 
with a smaller number.  The sequence in the encyclopedia is defined by ``Reading from right to left in the binary expansion of n, 
the number of 1's always stays ahead of the number of 0's''.

\section{Home Grown Theory}

\begin{theorem}
An integer $k$ with a glide with $n$ divides will also be a glide for $2^n+k$.
\end{theorem}

\begin{proof}
Suppose that an integer $k$ has a glide $G$ with $n$ dividing iterations and $m$ multiplication iterations.  By definition, 
$Gk=k_1<k$.  
Note that for $k$, the following holds since the multiplication operator results in a number always strictly greater 
than $3$ times the original:
\[
\frac{3^m}{2^n}k<k_1<k\textrm{.}
\]
Thus it must be that $3^m<2^n$.

Consider the number $p 2^n+k$.  The collatz operator applied to $p 2^n+k$ will result in the glide $G$ and 
\[
G(p2^n+k)=3^m+k_1\mbox{.}
\]
Since $p 3^m<p 2^n$ and $k_1<k$, it must be that $p 3^m+k_1<p 2^n+k$.
\end{proof}
\end{document}
